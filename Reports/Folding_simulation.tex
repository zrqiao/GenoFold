% !Mode:: "TeX:UTF-8"

\documentclass[11pt, a4paper]{article}
%\usepackage{xltxtra,fontspec,xunicode}
\usepackage{amsmath}
\usepackage{amssymb}
\usepackage{breqn}
\usepackage{autobreak}
\usepackage{braket,mleftright}
\usepackage{amsfonts}
\usepackage[section]{placeins}
\usepackage{float}
\usepackage{siunitx}

\usepackage{indentfirst}
\usepackage{caption}
\usepackage{physics}
\usepackage{geometry}
\usepackage{graphicx}
\usepackage{algorithm}% http://ctan.org/pkg/algorithm
\usepackage{algpseudocode}% http://ctan.org/pkg/algorithmicx

\geometry{top=1in, bottom=1in, left=1in, right=1in}
\linespread{1.5}

\DeclareMathOperator*{\argmax}{argmax}
\DeclareMathOperator*{\argmin}{argmin}

\newcommand{\degc}{$\,^\circ$C}

\begin{document}

\title{Simulate co-transcriptional folding kinetics by genetic algorithm}
\author{Zhuoran Qiao}
\date{\today}

\maketitle

We developed a genetic algorithm based approach to simulate kinetics of co-transcriptional folding. Our method is built on two following assumptions:
\paragraph{1} Most of RNA secondary structures (SS) are linkage of locally optimal or sub-optimal structures (foldons) at various folding sites;
\paragraph{2} Global structural rearrangement of a partial RNA segment is permitted only if it's folding to the optimal SS on that segment.

Formally, we denote direct product... irreducible foldon representation... green's function...

\section{Algorithm procedure}

During every elongation step, an active species pool of strands with unique SS and diffrent population is updated. New candidate strands with length of $L+\Delta L$ are generated by a recombination
process: for every old strand, all indices in its IFR is identified as possible rearrangement site, then its child strands is generated by linking partial segments
with a foldon that terminated at $L+\Delta L$. We assume that elongation will not change the inital population distribution of secondary structures: child strands with the exact
parental SS on $[0, L]$ will also inherit the population of their parents.

After structual generation the rate matrix among all candidate strands within the new active species pool is calculated (see part \ref{section:rate}). Then the population
distribution of strands after elongation is computed by chemical master equation, and one iterative elongation step is finished.

Pseudocodes of the whole procedure are as follows:

\begin{algorithm}
  \begin{algorithmic}[1]
    \State Initalize active pool
    \While{$\text{sequence length} > \text{current length }$}
      \State $\text{old pool} \gets \text{active pool }$
      \State $\text{renew active pool }$
      \State $\text{current length} += dL$
      \State $\text{dt} \gets \text{dL / \text{transcription rate}}$
      \For{$\text{left boundary} \in \text{\{0, dL, 2dL, ..., current length - dL\}}$}\Comment{Get all new foldons}
        \State $D^{foldon}(\text{left boundary}, \text{ current length}) \gets \text{numpy.mfe(sequence[left boundary, current length]})$
      \EndFor
      \For{$\text{strand} \in \text{old pool }$}\Comment{Recombination}
        \For{$\text{rearrangement site} \in \text{strand.IFR}$}
          \State $\text{Candidate} \gets$\\
          $D^{strand}(0, \text{ rearrangement site}) \oplus D^{foldon}(\text{rearrangement site}, \text{ current length})$
        \EndFor
      \EndFor
    \EndWhile
  \end{algorithmic}
\caption{Co-transcriptional folding elongation procedure}
\end{algorithm}

\begin{algorithm}
  \caption{Euclid’s algorithm}\label{euclid}
  \begin{algorithmic}[1]
    \Procedure{Euclid}{$a,b$}\Comment{The g.c.d. of a and b}
      \State $r\gets a\bmod b$
      \While{$r\not=0$}\Comment{We have the answer if r is 0}
        \State $a\gets b$
        \State $b\gets r$
        \State $r\gets a\bmod b$
      \EndWhile\label{euclidendwhile}
      \For{\texttt{<some condition>}}
        \State \texttt{<do stuff>}
      \EndFor
      \State \textbf{return} $b$\Comment{The gcd is b}
    \EndProcedure
  \end{algorithmic}
\end{algorithm}

\section{Secondary structure generation}
%\paragraph{a.} Derived formalisms for transition path statistics based on Transition Path Theory and continuous-time random walk (CTRW) model;
%\paragraph{b.} Suggested an simple method to analytically predict transition path time (TPT) distribution on arbitrary 1-d free energy landscapes (FELs);
%\paragraph{c.} Calculated probability distribution of backward jumping within transition paths ensemble for 1-d toy landscapes.

%\section{Progress}
%\subsection{Transition path statistics of CTRW model}

\paragraph{}Our goal is to figure out a general approach to calculate transition path statistics (average transit time and transit time distribution)
 for free energy landscape with discrete states. The Markov rate matrix $\mathbf{W}$ element between state $\sigma_i$ and state $\sigma_j$ is given by

 \begin{equation}
   \mathbf{W}_{ij}=\langle \sigma_i | \mathbf{W} | \sigma_j \rangle = k_0 \exp(-\frac{F_j-F_i}{2})
 \end{equation}

\begin{equation}
  \begin{split}
    P(t)&=\sum_{\varphi} P(t|\varphi)P(\varphi)\\
        &=\sum_{\varphi} P(t|\vec{H}[\varphi])P(\varphi)\\
        &=\sum_{\vec{H}} P(t|\vec{H})P(\vec{H})\\
        &=\sum_{\vec{H},l} P(t|\vec{H})P(\vec{H}|l)P(l)
  \end{split}
\end{equation}


\section{Folding pathway identification \& Rate calculation} \label{section:rate}

\paragraph{a.} Using a genetic-algorithm based method and NUPACK to predict populated RNA configurations during cotranscriptional folding (in progress);
\paragraph{b.} Using master equation method to simulate evolution of folding configurations and SD sequence accessibility.

%In order to examine when the one-dimensional coordinate projection could be recognized as effective reaction coordinate,
%we then examine the probability distribution of committors for transition path trajectories $p(q|TP)$, for which a single peak of probability $p(q|TP)$ have been ultilized as a indicator for 'good' reaction coordinates.
% Firstly, We found that for harmonic toy model, the shape of $p(q|TP)$ is very sensitive to the definition of source/sink region. For illustration, $p(q|TP)$ for two different selection
%  of source/sink regions was compared: in the first case, only two free energy minimum was indentified as source or sink\textbf{(S1)}; in the second case, only the barrier top was
%  defined as the transition path, while other two region of the free energy landscape was calssified as source/sink\textbf{(S2)}.

%\cite{Jacobs2018,Chaudhury2010,Vanden-Eijnden2010,Metzner,Krivov}

\small
%\bibliographystyle{plain}
%\bibliography{0727}

\end{document}
